\documentclass[12pt,english,pointednumbers,liststotoc,idxtotoc]{scrreprt}
\usepackage[T1]{fontenc}
\usepackage[latin9]{inputenc}
\usepackage[a4paper]{geometry}
\geometry{verbose,tmargin=25mm,bmargin=30mm,lmargin=25mm,rmargin=25mm}
\usepackage{fancyhdr}
\pagestyle{fancy}
\setcounter{secnumdepth}{3}
\setcounter{tocdepth}{3}
\usepackage{array}
\usepackage{verbatim}
\usepackage{makeidx}
\makeindex
\usepackage{graphicx}
\usepackage{setspace}
\usepackage{xcolor}
\usepackage{multirow}
\usepackage[most]{tcolorbox}
\definecolor{consolecolor}{RGB}{220,220,220}
\newcommand{\consolebox}[1]{\par\hspace{0.25cm}\colorbox{consolecolor}
	{\texttt{#1}}\newline}
\providecommand{\gitversion}[1]{1.00}
\makeatletter

%%%%%%%%%%%%%%%%%%%%%%%%%%%%%% LyX specific LaTeX commands.
%% Because html converters don't know tabularnewline
\providecommand{\tabularnewline}{\\}

%%%%%%%%%%%%%%%%%%%%%%%%%%%%%% User specified LaTeX commands.
% [From the LyX User Guide]...

\usepackage{ifpdf} % part of the hyperref bundle
\ifpdf % if pdflatex is used

% set fonts for nicer pdf view
\IfFileExists{lmodern.sty}{\usepackage{lmodern}}{}

% link all cross references and URLs in pdf output
\usepackage[colorlinks=true, bookmarks, bookmarksnumbered,
linkcolor=black, citecolor=black, urlcolor=blue, filecolor=blue,
pdfpagelayout=OneColumn, pdfnewwindow=true,
pdfstartview=XYZ, plainpages=false, pdfpagelabels,
pdfauthor={Gundolf Kiefer}, pdftex,
pdftitle={ParaNut},pdfsubject={ParaNut},
pdfkeywords={ParaNut}]{hyperref}

\else % if dvi or ps is produced

% link all cross references and URLs in dvi output
\usepackage[ps2pdf]{hyperref}

\fi % end if pdflatex is used

% the pages of the TOC are numbered roman
% and a pdf-bookmark for the TOC is added
\pagenumbering{roman}
\let\myTOC\tableofcontents
\renewcommand\tableofcontents{%
	\pdfbookmark[1]{Contents}{}
	\myTOC
	\cleardoublepage
	\pagenumbering{arabic} }

% define a short command for \textvisiblespace
\newcommand{\spce}{\textvisiblespace}

% redefine the greyed out note
%\renewenvironment{lyxgreyedout}
% {\textcolor{blue}\bgroup}{\egroup}


% [GK] ...

% Headings and footers...
\fancyhead[L]{\slshape \leftmark}
\fancyhead[C,R]{}
\fancyfoot[L]{\paranut: Configuration tool, User manual, \today}
\fancyfoot[C]{}
\fancyfoot[R]{\thepage}

\renewcommand{\footrulewidth}{0.6pt}
\renewcommand{\headrulewidth}{0.6pt}

\fancypagestyle{plain}{
	\fancyhead[L,C,R]{}
	\renewcommand{\headrulewidth}{0pt}
} 

%\newcolumntype{d}[1]{D{.}{.}{#1}}
% New column types to use in tabular environment for instruction formats.
% Allocate 0.18in per bit.
\newcolumntype{I}{>{\centering\arraybackslash}p{0.18in}}
% Two-bit centered column.
\newcolumntype{W}{>{\centering\arraybackslash}p{0.36in}}
% Three-bit centered column.
\newcolumntype{F}{>{\centering\arraybackslash}p{0.54in}}
% Four-bit centered column.
\newcolumntype{Y}{>{\centering\arraybackslash}p{0.72in}}
% Five-bit centered column.
\newcolumntype{R}{>{\centering\arraybackslash}p{0.9in}}
% Six-bit centered column.
\newcolumntype{S}{>{\centering\arraybackslash}p{1.08in}}
% Seven-bit centered column.
\newcolumntype{O}{>{\centering\arraybackslash}p{1.26in}}
% Eight-bit centered column.
\newcolumntype{E}{>{\centering\arraybackslash}p{1.44in}}
% Ten-bit centered column.
\newcolumntype{T}{>{\centering\arraybackslash}p{1.8in}}
% Twelve-bit centered column.
\newcolumntype{M}{>{\centering\arraybackslash}p{2.2in}}
% Sixteen-bit centered column.
\newcolumntype{K}{>{\centering\arraybackslash}p{2.88in}}
% Twenty-bit centered column.
\newcolumntype{U}{>{\centering\arraybackslash}p{3.6in}}
% Twenty-bit centered column.
\newcolumntype{L}{>{\centering\arraybackslash}p{3.6in}}
% Twenty-five-bit centered column.
\newcolumntype{J}{>{\centering\arraybackslash}p{4.5in}}

\tcbset{
    colback=gray!15,
	width=0.94\textwidth,
	colframe=gray!25,
	arc=0mm,
    }
\newenvironment{commentary}
{ 	\begin{center}
	\begin{tcolorbox}
	\small \em
		
	}
	{	
	\end{tcolorbox}
	\end{center}
}

\newcommand{\wunits}[2]{\mbox{#1\,#2}}

%\usepackage{graphicx} % part of the hyperref bundle
\newcommand{\paranut}{\mbox{\em ParaNut}}

\renewcommand{\caplabelfont}{\bf}

\makeatother

\usepackage{float}
\usepackage{babel}
\usepackage{listings}
\usepackage{xcolor}
\renewcommand{\lstlistingname}{Listing}
\definecolor{lstBackground}{RGB}{220,220,220}
\definecolor{colKeys}{rgb}{0,0,1}
\definecolor{colIdentifier}{rgb}{0,0,0}
\definecolor{colComments}{rgb}{1,0,0}
\definecolor{colString}{rgb}{0,0.5,0}
\lstset{
	float=hbp,
	basicstyle=\ttfamily\color{black}\small,
	identifierstyle=\color{colIdentifier},
	keywordstyle=\color{colKeys},
	stringstyle=\color{colString},
	commentstyle=\color{colComments},
	columns=flexible,
	tabsize=2,
	frame=single,
	extendedchars=true,
	showspaces=false,
	showstringspaces=false,
	numbers=left,
	numberstyle=\tiny,
	breaklines=true,
	backgroundcolor=\color{lstBackground},
	breakautoindent=true
}

\newcommand{\instbit}[1]{\mbox{\scriptsize #1}}
\newcommand{\instbitrange}[2]{~\instbit{#1} \hfill \instbit{#2}~}

